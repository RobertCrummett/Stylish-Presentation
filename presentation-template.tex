\documentclass[dark]{cgem-presentation}

%------------------------------------------
% Link to figures directory
\graphicspath{{./Figure}{./Logo}}

% Set bibliography directory
\bibliographypath{Bib}

%------------------------------------------
%% Title
% Affilations, Meeting information, etc
\newcommand{\cgemTitle}{Center for Gravity, Electrical \\& Magnetic Studies (CGEM)}
\newcommand{\csmTitle}{Department of Geophysics --- Colorado School of Mines}
\newcommand{\meetingTitle}{Conference --- Meeting Name --- Location}
\newcommand{\sessionTitle}{Session Information (if available)}
\newcommand{\daytimeTitle}{Date --- Time}
\newcommand{\cgemQuestion}{Center for Gravity, Electrical \\& Magnetic Studies}
\newcommand{\csmQuestion}{Colorado School of Mines}

% Title frame specifications
\title{Do Not Read Title to Audience}
\author{R. Nate Crummett*;Yaoguo Li}
\affiliations{\cgemTitle;\csmTitle}{\cgemQuestion;\csmQuestion}
\meeting{\meetingTitle;\sessionTitle;\daytimeTitle}
\contactInfo{robert\\_crummett@mines.com}

%------------------------------------------
%% Presentation
\begin{document}

  % Title slide
  \begin{frame}[plain]
    % Mines logo
    \begin{figure}
      \begin{textblock*}{2.05cm}(0.06\paperwidth,0.37\paperheight)
        \includegraphics[width=2.6cm]{\minesLogoFile}
      \end{textblock*}
    \end{figure}
    % CGEM logo
    \begin{figure}
      \begin{textblock*}{2.85cm}(0.77\paperwidth,0.35\paperheight)
        \includegraphics[width=2.85cm]{\cgemLogoFile}
      \end{textblock*}
    \end{figure}
    % Title
    \maketitle{1cm}{1.2cm}{2mm}{-5mm}
  \end{frame}

  % Adding outline slide
  \begin{frame}{Outline}
    \begin{enumerate}
      \item Often a Waste of Your Time
      \item Audiences Attention Span Wanes
      \item Be Meaningful! \\[1mm]
      \begin{enumerate}
        \item Introduce after hook \\[2mm]
        \begin{enumerate}
          \item Follow through talk (?)
        \end{enumerate}
      \end{enumerate}
    \end{enumerate}
  \end{frame}
 
  % Adding title to slide
  \begin{frame}{All Slides Should Be Titled}
    \LARGE 
    Hello world!
  \end{frame}

  % Add equations to slides
  \begin{frame}{Adding Equations}
    \LARGE
    Reduction To Pole (RTP) is a \textcolor{SecondColor}{rotation} \\[5mm]
    
    % RTP operator
    \begin{equation*}
      \widetilde{\mathbf{T}} =
        \textcolor{SecondColor}
        {
          \frac{(\hat{\mathit{u}} \cdot \vec{\mathit{k}})
          (\hat{\mathit{v}} \cdot \vec{\mathit{k}})}
          {\mathit{p}^2 + \mathit{q}^2}
        }
        \hspace{1mm}
        \widetilde{\mathbf{R}}
      \logvar{u}{$\\hat{\\mathit{u}}$}{Inducing magnetic field direction unit vector}{}
      \logvar{v}{$\\hat{\\mathit{v}}$}{Induced magnetization direction unit vector}{}
      \logvar{k}{$\\vec{\\mathit{k}}$}{Gradient operator in wavenumber domain}{}
      \logvar{p}{$\\mathit{p}$}{Wavenumber in X direction}{$\\text{m}^{-1}$}
      \logvar{q}{$\\mathit{q}$}{Wavenumber in Y direction}{$\\text{m}^{-1}$}
      \logvar{t}{$\\widetilde{\\mathbf{T}}$}{Observed total-field anomaly in the wavenumber domain}{nT}
      \logvar{r}{$\\widetilde{\\mathbf{R}}$}{Reduction to Pole (RTP) in the wavenumber domain}{nT}
    \end{equation*}
   
    \vspace{1cm}
    Fields are \textcolor{SecondColor}{rotated} as if they were vertical (ie, Pole)
  \end{frame}
  
  % Adding citations
  \begin{frame}{Citations Managed by Bib\TeX}
    \LARGE
    Everything gets hyperlinked! \cite{doi:10.1190/1.1444302} \\[2mm]

    Just use the \texttt{cite} function \cite{doi:10.1190/image2022-3729385.1} \\[2mm]
    
    Bib\TeX\ handles everything else \\[1cm]
    
    Do not like this format?

    Try another citation style! \cite{doi:10.1190/geo2020-0729.1}
  \end{frame}

  % Adding & citing figures
  \begin{frame}{Figure Demo --- Citations}
    \begin{figure}
      \begin{textblock*}{0.7\paperwidth}(0.15\paperwidth,0.18\paperheight)
        % Graphic
        \includegraphics[width=0.7\paperwidth]{weihermann\_et\_al\_2022\_inverted}
        % Citation
        \citefig{WEIHERMANN2021104682}
      \end{textblock*}
    \end{figure}
  \end{frame}

  % GMT figures, parameters in gmt.conf file to play with
  \begin{frame}
    \begin{figure}
      \begin{textblock*}{0.7\paperwidth}(-0.08\paperwidth,0.05\paperheight)
        % Graphic
        \includegraphics[width=0.5\paperwidth]{mag}
      \end{textblock*}
    \end{figure}
    \begin{textblock*}{0.5\paperwidth}(0.55\paperwidth,0.05\paperheight)
      \Huge
      \,\,\,Ideas for GMT
      \begin{enumerate}
        \item PS\_PAGE\_COLOR
        \item MAP\_FRAME\_PEN
        \item MAP\_DEFAULT\_PEN
        \item FONT\_TITLE
        \item FONT\_ANNOT
        \item FONT\_LABEL
        \item -\hspace{0.1mm}-PARAM //
          (override *.conf)
      \end{enumerate}
    \end{textblock*}

  \end{frame}

  % Acknowledgements
  \begin{frame}{Acknowledgments}
    \vspace{-2cm}
    \begin{enumerate}
      \item My sincere appreciation goes out to
      \begin{enumerate}
        \item GMRC
        \item Such and such at such and such
        \item So and so at so and so
        \item CGEM 
      \end{enumerate}
    \end{enumerate}
  \end{frame}

  % Question slide
  \begin{frame}[plain]
    % Mines logo
    \begin{figure}
      \begin{textblock*}{1cm}(0.05\paperwidth,0.025\paperheight)
        \includegraphics[width=1.25cm]{\minesLogoFile}
      \end{textblock*}
    \end{figure}
    % CGEM logo
    \begin{figure}
      \begin{textblock*}{1cm}(0.867\paperwidth,0.02\paperheight)
        \includegraphics[width=1.35cm]{\cgemLogoFile}
      \end{textblock*}
    \end{figure}
    % Question slide text
    \makequestion{-8mm}{-2mm}{1.75cm}{1.82cm}{1mm}
  \end{frame}

  \BibliographyFrame
  \VariableFrame

\end{document}
